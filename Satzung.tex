\documentclass[a4paper, 12pt]{scrartcl}
\usepackage[utf8]{inputenc}
\usepackage[T1]{fontenc}
\usepackage[ngerman]{babel}
\usepackage[official]{eurosym}
\usepackage{breakurl}
\usepackage[defaultfam]{montserrat}
\usepackage[autostyle]{csquotes}

\pagestyle{plain}
\title{Satzung}
\subtitle{ctrl.alt.coop eG}
\author{}
\date{11.04.2018}

% § for Section anzeigen
\renewcommand*\thesection{\S~\arabic{section}}

\urlstyle{same}

\KOMAoptions{toc=flat}
\begin{document}
\maketitle
\sffamily
\tableofcontents

\newpage
\section{Name, Sitz}
\begin{enumerate}
  \item Die Genossenschaft heißt \enquote{ctrl.alt.coop eG}.
  \item Der Sitz der Genossenschaft ist Schöneiche bei Berlin.
\end{enumerate}

\section{Zweck und Gegenstand}
\begin{enumerate}
  \item Die Genossenschaft bezweckt die Förderung der Wirtschaft und des Erwerbs der Mitglieder mittels gemeinschaftlichen Geschäftsbetriebes.
  \item Der Gegenstand der Genossenschaft ist:
  \begin{enumerate}
    \item die Entwicklung sowie Pflege und Wartung von IT‐Verfahren zur Unterstützung der Mitglieder bei der Aufgabenerfüllung sowie die Mitwirkung bei deren Einführung und Anwendung;
    \item Bereitstellung von IT‐Diensten einschließlich der Erbringung informationstechnischer und beratender Dienstleistungen für Mitglieder.
  \end{enumerate}
  \item Die Geschäfte mit Nichtmitgliedern sind zulässig.
  \item Die Genossenschaft kann sich an anderen Unternehmen beteiligen, wenn dies der Förderung der Mitglieder dient.
\end{enumerate}

\section{Mitgliedschaft}
\begin{enumerate}
  \item Zum Erwerb der Mitgliedschaft bedarf es einer unbedingten schriftlichen Beitrittserklärung, über die der Vorstand entscheidet.
  \item Mitglieder können werden:
    \begin{enumerate}
      \item Beschäftigte der Genossenschaft
      \item alle, die zur Gründung der Genossenschaft beigetragen haben.
    \end{enumerate}
  \item Die Mitgliedschaft endet durch:
  \begin {enumerate}
    \item Kündigung (Kündigung),
    \item Übertragung des gesamten Geschäftsguthabens,
    \item Tod, bzw. Auflösung einer juristischen Person oder einer Personengesellschaft oder
    \item Ausschluss (Ausschluss).
  \end{enumerate}
\end{enumerate}

\section{Geschäftsanteil}
\begin{enumerate}
  \item Der Geschäftsanteil beträgt 500 \euro. Er ist sofort in voller Höhe einzuzahlen.
  \item Jedes Mitglied darf sich mit mehr als einem Geschäftsanteil beteiligen. Über die Zulassung entscheiden Vorstand und die bevollmächtigte Person der Generalversammlung gemeinsam.
  \item Die Beteiligung eines Mitglieds mit einem zweiten Geschäftsanteil darf mit Ausnahme bei einer Pflichtbeteiligung erst zugelassen werden, wenn der erste Geschäftsanteil voll eingezahlt ist; das gleiche gilt für Beteiligung mit weiteren Geschäftsanteilen.
\end{enumerate}

\section{Ausschluss der Nachschusspflicht}
\begin{enumerate}
  \item Die Mitglieder sind nicht zur Leistung von Nachschüssen verpflichtet.
\end{enumerate}

\section{Rechte der Mitglieder}
\begin{enumerate}
  \item Alle Mitglieder haben, unabhängig von ihren Geschäftsanteilen, eine Stimme in der Generalversammlung und der Online-Generalversammlung.
  \item Jedes Mitglied hat das Recht, im Rahmen des Gesetzes die Leistungen in Anspruch zu nehmen und an der Gestaltung der Genossenschaft mitzuwirken.
  \item Alle Mitglieder haben das Recht auf der Generalversammlung Einsicht in das zusammengefasste Prüfungsergebnis zu nehmen.
  \item Alle Mitglieder haben das Recht das Protokoll der Generalversammlung einzusehen.
  \item Alle Mitglieder haben das Recht die Mitgliederliste einzusehen.
\end{enumerate}

\section{Pflichten der Mitglieder}
\begin{enumerate}
  \item Jedes Mitglied verpflichtet sich, die auf den Geschäftsanteil vorgeschriebenen Einzahlungen zu leisten.
  \item Jedes Mitglied ist verpflichtet die Interessen der Genossenschaft zu wahren.
  \item Jedes Mitglied hat den Bestimmungen des Genossenschaftsgesetzes, der Satzung und den Beschlüssen der Organe nachzukommen.
  \item Alle Mitglieder haben die Pflicht innerhalb von drei Wochen eine Änderung ihrer Anschrift dem Vorstand mitzuteilen.
  \item Das Mitglied ist verpflichtet, interne Informationen, Vorgänge oder sonstige Dinge, die der Genossenschaft erheblichen Schaden können zufügen können, nicht an unbeteiligte Dritte weiterzugeben.
\end{enumerate}

\section{Kündigung}
\begin{enumerate}
  \item Die Frist für die Kündigung der Mitgliedschaft oder einzelner, freiwilliger Anteile beträgt 6 Monate zum Schluss des Geschäftsjahres.
  \item Die Kündigung bedarf der Schriftform.
\end{enumerate}

\section{Übertragung des Geschäftsguthabens}
\begin{enumerate}
  \item Jedes Mitglied kann sein Geschäftsguthaben jederzeit durch schriftliche Vereinbarung einem anderen ganz oder teilweise übertragen und hierdurch seine Mitgliedschaft ohne Auseinandersetzung beenden oder die Anzahl seiner Geschäftsanteile verringern, sofern der Erwerber Mitglied der Genossenschaft wird oder bereits ist und das zu übertragende Geschäftsguthaben zusammen mit dem bisherigen Geschäftsguthaben den Gesamtbetrag der Geschäftsanteile, mit denen der Erwerber beteiligt ist oder sich zulässig beteiligt, nicht überschritten wird.
  \item Die Übertragung des Geschäftsguthabens bedarf der Zustimmung des Vorstandes.
\end{enumerate}

\section{Tod / Auflösung einer juristischen Person oder Personengesellschaft}
\begin{enumerate}
  \item Mit dem Tod eines Mitglieds geht die Mitgliedschaft auf den Erben über. Sie endet mit dem Schluss des Geschäftsjahres, in dem der Erbfall eingetreten ist.
  \item Wird eine juristische Person oder eine Personengesellschaft aufgelöst oder erlischt sie, so endet die Mitgliedschaft mit dem Schluss des Geschäftsjahres, in dem die Auflösung oder das Erlöschen wirksam geworden ist. Im Falle der Gesamtrechtsnachfolge wird die Mitgliedschaft bis zum Schluss des Geschäftsjahres durch den Gesamtrechtsnachfolger fortgesetzt.
\end{enumerate}

\section{Ausschluss}
\begin{enumerate}
  \item Mitglieder können zum Schluss eines Geschäftsjahres ausgeschlossen werden, wenn
  \begin{enumerate}
    \item sie die Genossenschaft schädigen,
    \item sie die gegenüber der Genossenschaft bestehenden Pflichten trotz Mahnung unter Androhung des Ausschlusses nicht erfüllen,
    \item sie unter der der Genossenschaft bekannt gegebenen Anschrift dauernd nicht erreichbar sind.
    \item wenn ein Angestelltenverhältnis mit einem Mitglied wegen schwerer Pflichtverletzungen gekündigt wird oder wurde;
    \item wenn es unrichtige oder unvollständige Erklärungen über seine rechtlichen und/oder wirtschaftlichen Verhältnisse abgibt;
    \item wenn es ein eigenes mit der Genossenschaft in Wettbewerb stehendes Unternehmen betreibt oder sich an einen solchen beteiligt, oder wenn ein mit der Genossenschaft in Wettbewerb stehendes Unternehmen sich an dem Unternehmen des Mitglieds beteiligt.
  \end{enumerate}
  \item Über den Ausschluss entscheidet die Generalversammlung unter Berücksichtigung der Mehrheitserfordernisse.
  \item Das Mitglied muss vorher angehört werden, es sei denn, dass der Aufenthalt eines Mitgliedes nicht ermittelt werden kann.
  \item Der Beschluss, durch den das Mitglied ausgeschlossen wird, ist dem Mitglied vom Vorstand unverzüglich durch eingeschriebenen Brief mitzuteilen.
  \item Das Mitglied verliert ab dem Zeitpunkt der Absendung der Mitteilung das Recht auf Teilnahme an der Generalversammlung oder der Vertreterversammlung sowie seine Mitgliedschaft im Vorstand, sein Amt als bevollmächtigte Person oder Revisor.
\end{enumerate}

\section{Auseinandersetzung}
\begin{enumerate}
  \item Das Ausscheiden aus der Genossenschaft hat die Auseinandersetzung zwischen dem ausgeschiedenen Mitglied bzw. dessen Erben und der Genossenschaft zur Folge. Die Auseinandersetzung unterbleibt im Falle der Übertragung von Geschäftsguthaben.
  \item Die Auseinandersetzung erfolgt aufgrund des von der Generalversammlung festgestellten Jahresabschlusses. Das nach der Auseinandersetzung sich ergebende Guthaben ist dem Mitglied binnen sechs Monaten nach seinem Ausscheiden auszuzahlen. Auf die Rücklagen und das sonstige Vermögen der Genossenschaft hat das ausgeschiedene Mitglied keinen Anspruch.
  \item Beim Auseinandersetzungsguthaben werden Verlustvorträge anteilig abgezogen.
  \item Wird die Genossenschaft binnen sechs Monaten nach dem Ausscheiden des Mitglieds aufgelöst, so gilt das Ausscheiden als nicht erfolgt.
\end{enumerate}

\section{Generalversammlung}
\begin{enumerate}
  \item Die Generalversammlung wird durch unmittelbare Benachrichtigung sämtlicher Mitglieder in Textform einberufen.
  \item Die Generalversammlung wird durch den Vorstand oder die bevollmächtigten Person der Generalversammlung einberufen.
  \item Die Einladung zur Generalversammlung muss mindestens zwei Wochen vor der Generalversammlung in Textform erfolgen. Bei der Einberufung ist die Tagesordnung bekannt zu machen. Ergänzungen der Beschlussgegenstände müssen den Mitgliedern mindestens eine Woche vor der Generalversammlung in Textform angekündigt werden. Die Mitteilungen gelten als zugegangen, wenn sie zwei Werktage vor Beginn der Frist abgesendet worden sind.
  \item Die Generalversammlung findet am Sitz der Genossenschaft statt, sofern nicht der Vorstand mit Zustimmung der bevollmächtigten Person der Generalversammlung einen anderen Ort festlegt.
  \item Jede ordnungsgemäß einberufene Generalversammlung ist unabhängig von der Zahl der Teilnehmer beschlussfähig.
  \item Jedes Mitglied hat eine Stimme.
  \item Die Mitglieder können Stimmrechtsvollmachten erteilen. Kein Bevollmächtigter darf mehr als zwei Mitglieder vertreten. Bevollmächtigte können nur Mitglieder der Genossenschaft, Ehegatten, eingetragene Lebenspartner, Eltern der Kinder eines Mitglieds oder Angestellte von juristischen Personen oder Personengesellschaften sein.
  \item Mehrheitserfordernisse für Abstimmungen sind in \$ 14 definiert.
  \item Die Generalversammlung bestimmt die Versammlungsleitung auf Vorschlag des Vorstands.
  \item Die Beschlüsse werden gem. § 47 GenG protokolliert.
\end{enumerate}

\section{Online-Generalversammlung}
\begin{enumerate}
  \item Die Generalversammlung kann auf der Website der Genossenschaft als Online-Generalversammlung abgehalten werden. Die Online-Generalversammlung besteht aus einer Diskussionsphase und einer anschließenden Abstimmungsphase.
  \item Mit der Einladung zur der Online-Generalversammlung erhalten die Mitglieder Zugangsdaten für die Teilnahme an der Diskussion und der Abstimmung, sowie den Beginn und das Ende der Diskussions- und Abstimmungsphase.
  \item Die Online-Generalversammlung wird von einem vom Vorstand bestimmten Mitglied des Vorstands geleitet.
  \item Die Diskussionen finden geschützt in einer geschlossenen Benutzergruppe statt. Zu jedem Tagesordnungspunkt werden Diskussionsbereiche eingerichtet, diese können vom Versammlungsleiter in Unterthemen gegliedert werden. Jedes Mitglied hat Diskussionsrecht. Anzahl und Umfang der Diskussionsbeiträge sind nicht beschränkt. Die Diskussionsphase dauert mindestens drei Wochen. Der Vorstand kann eine längere Diskussionsphase festlegen.
  \item Die Abstimmungsphase hat eine Dauer von sieben Tagen. Die Abstimmung erfolgt offen und namentlich. Die Abgabe einer Stimme erfolgt durch ein elektronisches Verfahren, das die Transparenz und Nachprüfbarkeit einer Stimmabgabe durch die Mitglieder sicherstellt. Das konkrete Abstimmungsverfahren wird vom Vorstand festgelegt. Jedes Mitglied kann bis drei Tage vor Beginn der Abstimmungsphase im Rahmen der angekündigten Beschlussgegenstände Anträge stellen und bereits gestellte eigene Anträge abändern oder zurückziehen. Der Versammlungsleiter entscheidet darüber ob über Anträge alternativ oder jeweils getrennt abgestimmt wird. Nach der Abstimmungsphase stellt der Versammlungsleiter unverzüglich das Abstimmungsergebnis fest und teilt es den Mitgliedern mit.
  \item Der Versammlungsleiter erstellt ein Protokoll der Online-Generalversammlung, das mindestens folgende Informationen enthält:
  \begin{enumerate}
    \item das Datum des Beginns der Diskussionsphase
    \item das Datum des Beginns und des Endes der Abstimmungsphase
    \item die Namen der Mitglieder, die an der Abstimmung teilgenommen haben,
    \item den Wortlaut der Anträge, die Abstimmungsergebnisse und Äußerungen, deren Aufnahme in das Protokoll ausdrücklich verlangt wurde.
  \end{enumerate}
  \item Das Protokoll wird vom Versammlungsleiter und allen Vorstandsmitgliedern unterschrieben und auf der Website der Genossenschaft im geschützten Mitgliederbereich veröffentlicht. Gegen das Protokoll kann innerhalb von sieben Tagen nach Veröffentlichung Einspruch erhoben werden.
\end{enumerate}

\section{Mehrheitserfordernisse}
\begin{enumerate}
  \item Es wird angestrebt Entscheidungen im Konsens zu treffen.
  \item Die Beschlüsse der Generalversammlung bedürfen der einfachen Mehrheit der gültigen abgegebenen Stimmen, soweit nicht das Gesetz oder diese Satzung eine größere Mehrheit vorschreibt.
  \item Eine Mehrheit von drei Vierteln der gültigen abgegebenen Stimmen ist in folgenden Fällen erforderlich:
  \begin{enumerate}
    \item Änderung der Satzung;
    \item Austritt aus genossenschaftlichen Verbänden;
    \item Verschmelzung, Spaltung und Formwechsel der Genossenschaft;
    \item Aufnahme, Übertragung oder Aufgabe eines wesentlichen Geschäftsbereichs,
    \item Auflösung der Genossenschaft;
    \item Fortsetzung der Genossenschaft nach beschlossener Auflösung;
    \item Ausschluss von Mitgliedern.
  \end{enumerate}
\end{enumerate}

\section{Bevollmächtigte Person, Revisionskommission}
\begin{enumerate}
  \item Es wird kein Aufsichtsrat gebildet.
  \item Die Generalversammlung wählt aus ihrer eine bevollmächtigte Person und bestimmt deren Amtszeit.
  \item Die Amtszeit der bevollmächtigten Person dauert fort bis zur ordentlichen Generalversammlung, die auf den formellen Ablauf der Amtszeit folgt.
  \item Die bevollmächtigte Person vertritt gemäß § 39 Abs. 1 S. 2 GenG die Genossenschaft gegenüber den Vorstandsmitgliedern im Rahmen der Beschlüsse der Generalversammlung und übernimmt gemäß § 57 Abs. 6 GenG im Rahmen der gesetzlichen Prüfung die Aufgaben, die ansonsten ein Aufsichtsratsvorsitzender gehabt hätte (gesetzliche Aufgaben).
  \item Zusätzlich übernimmt sie nach § 38 Absatz 1 Satz 3 in Verbindung mit § 9 Absatz 1 Satz 3 GenG die Prüfung des Jahresabschlusses und die Kenntnisnahme des Prüfungsberichtes und berichtet der Generalversammlung über die Ergebnisse; die grundsätzliche Verantwortlichkeit der Generalversammlung bleibt hiervon unberührt.
  \item Die Generalversammlung kann zur Unterstützung der bevollmächtigten Person bei der Erfüllung der Aufgaben nach Absatz 3 Revisoren wählen. Die Amtszeit der Revisoren entspricht der Amtszeit der gewählten bevollmächtigten Person.
  \item Die bevollmächtigte Person ist befugt, nach freiem Ermessen Mitglieder des Vorstands vorläufig, bis zur Entscheidung der unverzüglich einzuberufenden Generalversammlung, von ihren Geschäften zu entheben.
\end{enumerate}

\section{Vorstand}
\begin{enumerate}
  \item Der Vorstand besteht aus mindestens einem Mitglied.
  \item Die Generalversammlung bestimmt die Anzahl, wählt die Mitglieder des Vorstands.
  \item Die Amtszeit dauert bis zur ordentlichen Generalversammlung drei Jahre nach der Wahl.
  \item Besteht der Vorstand aus mehreren Mitgliedern, kann er auch schriftlich, telefonisch und auf elektronischem Wege Beschlüsse fassen, wenn kein Vorstandsmitglied diesem Weg der Beschlussfassung widerspricht.
  \item Die Genossenschaft wird durch zwei Vorstandsmitglieder vertreten. Hat die Genossenschaft nur ein Vorstandsmitglied, vertritt dieses die Genossenschaft alleine.
  \item Der Vorstand führt die Genossenschaft in eigener Verantwortung. Er bedarf der Zustimmung des Generalversammlung für
  \begin{enumerate}
    \item Investitionen oder Aufnahme von Krediten,
    \item Abschlüsse von Miet-, Pacht- oder Leasingverträgen, sowie anderen Verträgen mit wiederkehrenden Verpflichtungen mit einer Laufzeit von mehr als 3 Jahren und/oder einer jährlichen Belastung von mehr als 2.000 \euro,
    \item die Errichtung und Schließung von Filialen,
    \item die Gründung von Unternehmen und die Beteiligung an anderen Unternehmen,
    \item das Auslagern von Aufgaben und Tätigkeiten an externe Dienstleister oder Tochtergesellschaften,
    \item sämtliche Grundstücksgeschäfte,
    \item Erteilung von Prokura und
    \item die Aufstellung und Änderung der Geschäftsordnung für den Vorstand.
  \end{enumerate}
  \item Der Vorstand hat mit der Generalversammlungen Wirtschafts- und Stellenplan zu beraten. Er hat der Generalversammlung mindestens vierteljährlich, auf Verlangen oder bei wichtigem Anlass unverzüglich, über die geschäftliche Entwicklung der Genossenschaft zu berichten. Dabei muss er auf Abweichungen vom Wirtschafts- und Stellenplan eingehen.
  \item Der Vorstand bedarf für die Aufnahme des 21. Mitgliedes der Zustimmung der Generalversammlung. Bei der Einladung zu dieser Generalversammlung soll der Vorstand Wahlen zum Aufsichtsrat und ggfs. Vorstand, sowie entsprechende Satzungsänderungen auf die Tagesordnung zu setzen.
  \item Der Vorstand kann vorzeitig nur von der Generalversammlung abberufen werden.
\end{enumerate}

\section{Gemeinsame Vorschriften für die Organe}
\begin{enumerate}
  \item Niemand kann für sich oder einen anderen das Stimmrecht ausüben, wenn darüber Beschluss gefasst wird, ob er oder das vertretene Mitglied zu entlasten oder von einer Verbindlichkeit zu befreien ist oder ob die Genossenschaft gegen ihn oder das vertretene Mitglied einen Anspruch geltend machen soll.
  \item Wird über Angelegenheiten der Genossenschaft beraten, die die Interessen eines Organmitglieds, seines Ehegatten, seiner Eltern, Kinder und Geschwister oder von ihm kraft Gesetzes oder Vollmacht vertretenen Person berühren, so darf das betreffende Mitglied an der Beratung nicht teilnehmen. Das Mitglied ist jedoch vor der Beschlussfassung zu hören.
\end{enumerate}

\section{Gewinnverteilung, Verlustdeckung, Rückvergütung und Rücklagen}
\begin{enumerate}
  \item Über den bei der Feststellung des Jahresabschlusses sich ergebenden Gewinn oder Verlust des Geschäftsjahres entscheidet die Generalversammlung innerhalb von sechs Monaten nach Schluss des Geschäftsjahres.
  \item Die Generalversammlung kann einen Verlust aus Rücklagen decken, auf neue Rechnung vortragen oder auf die Mitglieder verteilen.
  \item Bei einem Gewinn kann die Generalversammlung nach Zuführung des erforderlichen Anteils in die gesetzliche Rücklage und der Verzinsung von Geschäftsguthaben den verbleibenden Gewinn in die freie Rücklage einstellen, auf neue Rechnung vortragen oder diesen an die Mitglieder verteilen.
  \item Die Verteilung von Verlust und Gewinn auf die Mitglieder geschieht im Verhältnis des Standes der Geschäftsguthaben am Schluss des vorhergegangenen Geschäftsjahres.
  \item Eine Auszahlung von Gewinnen erfolgt erst bei vollständig aufgefüllten Geschäftsguthaben.
  \item Der gesetzlichen Rücklage sind mindestens 20\% des Jahresüberschusses zuzuführen, bis mindestens 100\% der Summe der Geschäftsanteile erreicht sind.
  \item Die Mitglieder haben Anspruch auf die vom Vorstand mit Zustimmung der Generalversammlung beschlossene Rückvergütung.
  \item Ansprüche auf Auszahlung von Gewinnen, Rückvergütungen und Auseinandersetzungsguthaben verjähren in zwei Jahren ab Fälligkeit. Die Beträge werden den Rücklagen zugeführt.
\end{enumerate}

\section{Bekanntmachungen}
\begin{enumerate}
  \item Bekanntmachungen, deren Veröffentlichung vorgeschrieben ist, erfolgen unter der Firma der Genossenschaft im elektronischen Bundesanzeiger und unter \url{www.genossenschaftsbekanntmachungen.de}.
\end{enumerate}


% bei updates eintragen?
%\section*{Änderungen}
%\begin{itemize}
%\end{itemize}
\end{document}
